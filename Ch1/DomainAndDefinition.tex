\lhead{\emph{\ChapterOne{}}}
\section{Ámbito del problema}

Se decidió realizar un trabajo sobre Procesamiento de Lenguaje Natural, siendo este el primer punto acordado antes de proceder a la definición del problema a tratar para la realización del trabajo.

\section{Definición del problema}

El caso de estudio escogido para este trabajo puede ser subdividido en dos problemas: el primero es la obtención del texto a clasificar, ya sea desde un archivo de texto en local o a través de Internet.

El segundo problema es el más importante y el cual es objetivo principal de este trabajo: el procesamiento del texto una vez obtenido.

Se ha decidido presentar las secciones de manera cronológicamente equivalente a la que se produce en cualquier problema de Procesamiento de Lenguaje Natural.

%% TODO : Does this go here?

%\subsection{Datos de trabajo}

\subsection{Obtención del texto}

Hoy en día existen muchas fuentes de las cuales obtener texto a la hora de realizar procesamiento de texto: se puede obtener de Internet, puede ser creado o incluso puede obtenerse de un medio físico como pueden ser los libros o la voz de un usuario.

La obtención del texto a procesar puede realizarse de muchas maneras, existiendo varias opciones para cada fuente de texto que se puede presentar. 

En el caso de Internet, existen tanto librerías para lenguajes de programación como programas dedicados a la extracción de texto. Normalmente funcionan de manera similar a un navegador de Internet, utilizando el protocolo HTTP para enviar peticiones a servidores web y recibir contenidos de las páginas web, que luego son desglosados mediante \textit{parsers} para obtener los datos deseados.

En nuestro caso, el objetivo es la clasificación de los Trabajos de Fin de Grado. El Repositorio Institucional de la Universidad de La Laguna \cite{riull} es un Servicio Web que contiene una gran cantidad de documentos relacionados con la actividad docente de la Universidad, entre ellos los Trabajos de Fin de Grado.

\subsection{Procesamiento del texto}

El procesamiento de texto es la parte del problema que toma una mayor importancia, ya que es la más complicada y objeto de este trabajo.

Para nuestro caso de estudio, nos concentramos en un subproblema del campo; la clasificación de textos. La clasificación de textos es una de las de mayor relevancia actual.