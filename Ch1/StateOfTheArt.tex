\section{Estado del arte}

El estado actual del campo tiene numerosas áreas temáticas que se pueden subclasificar en herramientas conceptuales y aplicaciones. Una lista extensa de éstas se define en el número 56 de la Revista de Procesamiento de Lenguaje Natural \cite{sepln-56}:
        
\textbf{Modelos lingüísticos, matemáticos y psicolingüísticos del lenguaje}, refiriéndonos al PLN enfocado como un sistema de reglas.

\textbf{Lingüística de corpus}, metodología en la que se estudia el lenguaje a través de ejemplos de textos reales producidos en el mundo real.

\textbf{Desarrollo de recursos y herramientas lingüísticas}, en aquéllos trabajos que expanden las herramientas PLN disponibles para su futuro uso.

\textbf{Gramáticas y formalismos para el análisis morfológico y sintáctico}, refiriéndonos al PLN enfocado como un sistema de reglas lógicas aplicado al análisis morfosintáctico.

\textbf{Semántica, pragmática y discurso}. El estudio tradicional del lenguaje natural para cada idioma por parte de lingüistas nos permite abstraer las reglas para su uso y comprensión.

\textbf{Lexicografía y terminología computacional}, que busca una sistemática colección y explicación de todas las palabras (o más estrictamente, unidades léxicas) de un lenguaje. 

\textbf{Resolución de la ambigüedad léxica}, siendo un ejemplo la determinación del sustantivo concreto dada una referencia indirecta.

\textbf{Aprendizaje automático en PLN}, es decir la aplicación de algoritmos de aprendizaje no supervisado para sistemas de esta índole.

\textbf{Generación textual monolingüe y multilingüe}. Esto es, la expresión en lenguaje natural de información estructurada, permitiendo a un ordenador generar a tiempo real respuestas más accesibles a un usuario.

\textbf{Traducción automática}, refiriéndonos a sistemas capaces de realizar la traducción de textos naturales a otro lenguaje natural con poca o nula supervisión humana.

\textbf{Reconocimiento y síntesis del habla}, atendiendo a problemática como la determinación de palabras homófonas a través del contexto.

\textbf{Extracción y recuperación de información monolingüe, multilingüe y multimodal}, en la que se aprovecha información multimedia recibida a tiempo real para analizar la intención del emisor.

\textbf{Sistemas de búsqueda de respuestas}, alternativas a los históricos sistemas expertos, en los que se pretende responder a preguntas del usuario sobre un ámbito reducido.

%\textbf{Análisis automático del contenido textual}.

\textbf{Resumen automático}, por el cual obtenemos el resumen de un texto intentando conservar la información relevante.

%\textbf{PLN para la generación de recursos educativos}.

\textbf{PLN para lenguas con recursos limitados}, como es el caso de lenguas indígenas o autóctonas.

\textbf{Aplicaciones industriales del PLN}, entre las más importantes la generación de información para la toma de decisiones en empresas, o Inteligencia Empresarial (\textit{Business Intelligence}).

\textbf{Sistemas de diálogo}, sistemas que se comunican con el usuario utilizando lenguaje natural, de manera similar a mantener una conversación. 

\textbf{Análisis de sentimientos y opiniones}, actualmente un área del PLN con gran popularidad y que puede considerarse como una de sus aplicaciones industriales de mayor interés.

\textbf{Minería de texto}, el procesamiento estadístico de grandes cantidades de texto con la intención de extraer información de alta calidad.

%\textbf{Evaluación de sistemas de PLN}.

\textbf{Implicación textual y paráfrasis}, con la que podemos visualizar información textual de diversas maneras, así como conocer las enunciaciones que se derivan de ellas.

Se tuvieron en cuenta estas áreas temáticas a la hora de decidir el enfoque que tendría el trabajo, tomando en cuenta variables como factibilidad, tiempo disponible y utilidad, entre otras.

