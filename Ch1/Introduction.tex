\lhead{\emph{\ChapterOne{}}}
El Procesamiento del Lenguaje Natural (PLN) es una disciplina de la Computación que concierne al tratamiento de información desestructurada, presentada en texto o audio y expresada mediante Lenguaje Natural.
%
Se entiende por lenguaje natural cualquier lenguaje utilizado para la comunicación humana en el día a día, en contraposición a un lenguaje artificial establecido, como por ejemplo los lenguajes de programación.
%
Es un área de interés para el campo de la Interacción Persona-Computador (\textit{Human-Computer Interaction}), así como los de Aprendizaje Automático (\textit{Machine Learning}) e Inteligencia Artificial (\textit{Artificial Intelligence}).

La necesidad del Procesamiento del Lenguaje Natural es aparente en los sistemas de Interacción Persona-Computador, dado que en este campo se busca el diseño de interfaces cómodas y funcionales para los usuarios humanos a la hora de utilizar ordenadores, ya sean tradicionales, portátiles, móviles, etc.
%
El Procesamiento del Lenguaje Natural se posiciona como una capacidad deseable en los ordenadores, para permitir la comunicación natural entre el usuario y el ordenador, y aumentar tanto la comodidad y conveniencia como las posibilidades de una interacción exitosa.

En cuanto al campo del Aprendizaje Automático, no sólo nos referimos al hecho de que éste campo es necesario para el desarrollo de técnicas PLN, sino que el campo del Procesamiento del Lenguaje Natural a su vez puede ser útil para el Aprendizaje Automático en uno de sus mayores problemas.
%
Uno de los principales cuellos de botella hoy en día a la hora de la construcción de sistemas de Aprendizaje Automático es la insuficiente cantidad de datos estructurados. El Procesamiento del Lenguaje Natural permite la conversión de los datos desestructurados en lenguaje natural en datos estructurados con los cuales un sistema informático pueda aprender.

Sólo por esto es por extensión de importancia para el campo de la Inteligencia Artificial, ya que los recientes avances en dicho campo se basan en el Aprendizaje Automático y por tanto se beneficia de las mejoras que el PLN pueda aportar al mismo.
%
No es necesario concretar la importancia del PLN en el campo de la Inteligencia Artificial, ya que las primeras definiciones de una máquina inteligente (el Test de Turing) presuponen una capacidad de procesamiento e interpretación del lenguaje humano.

En otros campos de la ciencia se ha hablado extensamente de la importancia del lenguaje como factor decisivo para la producción de inteligencia tal y como la definimos, tanto así que existe una rama de la ciencia, la neurolinguística, centrada exclusivamente en el estudio de este fenómeno.