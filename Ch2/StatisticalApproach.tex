Esta clase de enfoques hacen uso de las técnicas de Aprendizaje Automático, que hoy en día son las que gozan de mayor aceptación y crecimiento.
%
Utilizan técnicas estadísticas para realizar estimaciones sobre un conjunto de datos del que se desconoce una o varias características, y se procede a asignar
%
El éxito de este tipo de técnicas se basa en las características que sean detectadas y seleccionadas como relevantes para el problema de clasificación que se tenga.

\begin{minipage}{\linewidth}
\textit{Tomemos como ejemplo un hipotético estudio sobre los hábitos alimenticios de un grupo de animales.
El color del pelaje de un animal no es relevante a la hora de inferir su dieta, pero en cambio la clase de dentadura que posea sí lo es. Por tanto, en este caso se deberá ignorar el color del pelaje pero tomar en cuenta el tipo de dentadura de los animales.}
\end{minipage}

%Por tanto, es 

Para los casos de aprendizaje supervisado, existen elementos ya clasificados de manera correcta, ya sea por parte de otro programa o de manera manual. Estos elementos son utilizados para conformar un subconjunto de entrenamiento.

En estos casos de aprendizaje supervisado, es importante cuidarse de no sobreajustar el clasificador a la clasificación de los problemas de entrada. 
%
El problema del sobreajuste es un problema difícil de resolver objetivamente, y suele manejarse mejor cuando se tiene un cierto nivel de experiencia en tratar con él.

En el caso del aprendizaje no supervisado, no se poseen de datos de entrenamiento, por lo que normalmente recae sobre el algoritmo la diferenciación de los elementos en diferentes clases \say{a posteriori}, es decir, clases que no habían sido definidas antes de la tarea de clasificación.

%Este método puede brindar nuevas maneras de...

%Uno de los problemas que tiene esta clase de enfoque es que muchas veces no proporciona un modelo claro, con algunas excepciones, de la lógica utilizada para clasificar los textos.

%Las excepciones a la regla son los árboles de clasificación, que pueden ser utilizados una vez se conoce la estructura del problema.
%Si se desconoce la estructura del problema pero en cambio se tienen datos etiquetados, pueden ser generados mediante técnicas de Aprendizaje Automático.