Los clasificadores de Bayes ingenuos (\textit{Naive Bayes classifiers}) son clasificadores utilizados ampliamente en la actualidad para problemas de clasificación de textos, tanto en ámbitos académicos como industriales.

Los caracteriza una gran simplicidad, que conlleva la ventaja de que son de fácil comprensión y por tanto de fácil uso.

A pesar de esta simplicidad, los clasificadores de Bayes ingenuos son bastante eficaces y tienen una gran cantidad de aplicaciones prácticas.

Los clasificadores de Bayes ingenuos se basan en un modelo probabilístico, el Teorema de Bayes.

\begin{equation}\label{eq:bayes}
    p(C_k|x) = \frac{p(C_k)p(x|C_k)}{p(x)}
\end{equation}

Según este teorema, la probabilidad de un suceso $C_k$, dada una situación $x$ viene determinada por el número de veces que se ha producido el suceso $C_k$ en esa situación $x$, dividido por el número de veces total que se ha encontrado esta situación $x$
(Eq. \ref{eq:bayes}).

Aplicados al problema de la clasificación, nos permite inferir la posibilidad que tiene un elemento de pertenecer a cada una de las clases consideradas, teniendo en cuenta cada una de las características de este elemento como un suceso a priori.

Los clasificadores de Bayes ingenuos reciben su nombre debido a que toman en cuenta la contribución de cada característica hacia la posibilidad de una clasificación de manera independiente, en contraposición a correlacionar varias características para realizar una estimación más precisa a la hora de clasificar un elemento.