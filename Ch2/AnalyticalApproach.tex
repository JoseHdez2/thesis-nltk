\lhead{\emph{\ChapterTwo{}}}
Los enfoques analíticos son aquéllos que intentan realizar un análisis morfosintáctico sobre el texto de entrada, obteniendo información relativa a las construcciones sintácticas de las oraciones y la morfología de las palabras utilizadas.
%
Esta clase de enfoques se basa en el análisis morfosintáctico clásico, que lleva realizándose durante siglos de forma manual, como herramienta docente para dar a conocer las bases de un lenguaje natural.

Sin embargo, se trata de enfoques de mayor complejidad en comparación con los enfoques estadísticos, debido a que requieren la implementación de conocimientos sobre la gramática, sintaxis y semántica del lenguaje objetivo.
%
Por ejemplo se deben de conocer todas las conjugaciones posibles de todos los verbos del lenguaje objetivo, y atender al contexto de las oraciones para utilizar la acepción correcta de palabras ambigüas.

Por otra parte, las herramientas de PLN deben contener corpus morfosintácticos y semánticos para cada uno de los idiomas, además de requerir para cada uno de ellos diferente para paliar las idiosincracias de cada uno de los lenguajes.

Debido a estas razones, y con motivo de atenerse al tiempo disponible, se ha decidido no adentrarnos en esta clase de análisis, que sin embargo resulta bastante prometedor.