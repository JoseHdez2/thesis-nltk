\lhead{\emph{\ChapterTwo{}}}
%\subsection{Almacenamiento de la información}
Se decidió guardar la información obtenida del proceso de extracción de texto en archivos locales en el disco duro del ordenador utilizado para la realización del trabajo.

El formato utilizado para los archivos de texto es CSV (\textit{Comma Separated Values}), es decir, archivos de texto planos-
%
Se trata de un formato sencillo en el cual cada elemento se encuentra en una línea diferente, y las características de dichos elementos separados por comas.

La razón por la que se utilizó éste formato se debe a la mayor comodidad que aporta en un caso de estudio restringido, con una cantidad de datos relativamente pequeña.
%
Otra razón por la que se eligió el formato CSV es la disponibilidad de librerías Python que trabajan en este formato, incrementando la facilidad y rapidez de desarrollo del sistema de almacenamiento.