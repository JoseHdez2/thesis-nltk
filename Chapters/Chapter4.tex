\chapter{\ChapterFour{}}
%\lhead{\emph{\ChapterFour{}}}

\section{Conclusiones y Trabajos Futuros}

Como líneas futuras de trabajo, se considera la extracción de texto de los Trabajos de Fin de Grado en formato infijo (extensión PDF), con el fin de obtener una cantidad suficiente de palabras para caracterizar cada una de las instancias a clasificar.

Por otra parte, se tiene en cuenta la posibilidad de descargar datos de TFGs de diferentes disciplinas a la Informática, para aumentar la diferenciación entre instancias y esperando que la capacidad de clasificación incremente de manera considerable.

Finalmente, se puede utilizar un clasificador estadístico diferente al clasificador bayesiano ingenuo, o también considerar otros métodos de definición y extracción de características.

---

Las técnicas de Procesamiento del Lenguaje Natural tienen hoy en día un gran número de aplicaciones prácticas, gracias a la disponibilidad de una gran cantidad de datos en Internet y a la creciente capacidad de procesamiento tradicional y en la nube (\textit{cloud computing}), dando soporte a las técnicas de aprendizaje estadístico que gozan de gran aceptación en la actualidad.

La herramienta que goza de mayor popularidad actualmente tanto para el Procesamiento del Lenguaje Natural como otras ramas de la Computación es el Aprendizaje Automático, gracias a la mayor cantidad de información y capacidad de procesamiento disponibles.

Esta herramienta ha permitido un incremento considerable de las aplicaciones prácticas del Procesamiento del Lenguaje Natural durante la última década. Esto ha creado nuevos puestos de trabajo en grandes compañías y pequeñas \textit{startups} de nueva creación que han decidido tomar la oportunidad que estas nuevas tecnologías brindan.

Esta inversión en el área del Procesamiento del Lenguaje Natural también se ve reflejada en el ámbito académico, y producirá durante las próximas décadas nuevas herramientas y aplicaciones para la sociedad, en sinergia con los campos de la Interacción Persona-Computador y la Inteligencia Artificial.

\section{Conclusions and Future Work}

For future lines of work, we will consider the extraction of the Final Degree Projects in infix format (PDF extension), in order to obtain a sufficient amount of words so as to characterize each of the instances that are to be classified.

We also take into account the possibility of downloading data from Final Degree Projects of fields different from Computer Science, in order to increase the differentiation between instances and in hopes of increasing the classification capacity considerably.

Finally, a classifier different from the Naive Bayes classifier can be used, or different methods of feature definition and extraction can be considered.

---

At present, Natural Language Processing techniques have a wide range of practical applications, thanks to the availability of a vast amount of data on the Internet and the increasing processing capabilities of computers, both locally and in cloud computing, supporting the statistical learning techniques that are widely accepted to date.

The tool which has garnered the most popularity in recent years, in the field of Natural Language Processing as well as other fields of Computer Science, is Machine Learning, thanks to the higher amount of information and computational power currently available.

This tool has allowed for a significant increase of the practical applications of Natural Language Processing during the last decade. This has created new jobs in large companies and newly-created \textit{startups} that have seized the opportunity that these new technologies bring.

This investment in the area of Natural Language Processing is also reflected in the academical world, and in the next decades it will provide new tools and applications for society, in synergy with the fields of Human-Computer Interaction and Artificial Intelligence.