\chapter{\ChapterOne{}}
\lhead{\emph{\ChapterOne{}}}

\lhead{\emph{\ChapterOne{}}}
El Procesamiento del Lenguaje Natural (PLN) es una disciplina de la Computación que concierne al tratamiento de información desestructurada, presentada en texto o audio y expresada mediante Lenguaje Natural.
%
Se entiende por lenguaje natural cualquier lenguaje utilizado para la comunicación humana en el día a día, en contraposición a un lenguaje artificial establecido, como por ejemplo los lenguajes de programación.
%
Es un área de interés para el campo de la Interacción Persona-Computador (\textit{Human-Computer Interaction}), así como los de Aprendizaje Automático (\textit{Machine Learning}) e Inteligencia Artificial (\textit{Artificial Intelligence}).

La necesidad del Procesamiento del Lenguaje Natural es aparente en los sistemas de Interacción Persona-Computador, dado que en este campo se busca el diseño de interfaces cómodas y funcionales para los usuarios humanos a la hora de utilizar ordenadores, ya sean tradicionales, portátiles, móviles, etc.
%
El Procesamiento del Lenguaje Natural se posiciona como una capacidad deseable en los ordenadores, para permitir la comunicación natural entre el usuario y el ordenador, y aumentar tanto la comodidad y conveniencia como las posibilidades de una interacción exitosa.

En cuanto al campo del Aprendizaje Automático, no sólo nos referimos al hecho de que éste campo es necesario para el desarrollo de técnicas PLN, sino que el campo del Procesamiento del Lenguaje Natural a su vez puede ser útil para el Aprendizaje Automático en uno de sus mayores problemas.
%
Uno de los principales cuellos de botella hoy en día a la hora de la construcción de sistemas de Aprendizaje Automático es la insuficiente cantidad de datos estructurados. El Procesamiento del Lenguaje Natural permite la conversión de los datos desestructurados en lenguaje natural en datos estructurados con los cuales un sistema informático pueda aprender.

Sólo por esto es por extensión de importancia para el campo de la Inteligencia Artificial, ya que los recientes avances en dicho campo se basan en el Aprendizaje Automático y por tanto se beneficia de las mejoras que el PLN pueda aportar al mismo.
%
No es necesario concretar la importancia del PLN en el campo de la Inteligencia Artificial, ya que las primeras definiciones de una máquina inteligente (el Test de Turing) presuponen una capacidad de procesamiento e interpretación del lenguaje humano.

En otros campos de la ciencia se ha hablado extensamente de la importancia del lenguaje como factor decisivo para la producción de inteligencia tal y como la definimos, tanto así que existe una rama de la ciencia, la neurolinguística, centrada exclusivamente en el estudio de este fenómeno.

---

Para facilitar la fluidez en la lectura del documento, se hace uso de términos y abreviaturas que pueden ser propias del área de conocimientos a tratar o simplemente sean de nueva introducción, por lo que se proporciona un glosario como apéndice auxiliar con el fin de esclarecer su significado.

En el siguiente apartado, se hace una revisión de las áreas temáticas y herramientas en las que se subdivide el campo del Procesamiento del Lenguaje Natural. El resto del Capítulo 1 está dedicado a las herramientas consideradas para la realización del caso de estudio.

\lhead{\emph{\ChapterOne{}}}
\section{Estado del arte}

El estado actual del campo tiene numerosas áreas temáticas que se pueden subclasificar en herramientas conceptuales y aplicaciones. Una lista extensa de éstas se define en el número 56 de la Revista de Procesamiento de Lenguaje Natural \cite{sepln-56}:
        
\textbf{Modelos lingüísticos, matemáticos y psicolingüísticos del lenguaje}, refiriéndonos al PLN enfocado como un sistema de reglas.

\textbf{Lingüística de corpus}, metodología en la que se estudia el lenguaje a través de ejemplos de textos reales producidos en el mundo real.

\textbf{Desarrollo de recursos y herramientas lingüísticas}, en aquéllos trabajos que expanden las herramientas PLN disponibles para su futuro uso.

\textbf{Gramáticas y formalismos para el análisis morfológico y sintáctico}, refiriéndonos al PLN enfocado como un sistema de reglas lógicas aplicado al análisis morfosintáctico.

\textbf{Semántica, pragmática y discurso}. El estudio tradicional del lenguaje natural para cada idioma por parte de lingüistas nos permite abstraer las reglas para su uso y comprensión.

\textbf{Lexicografía y terminología computacional}, que busca una sistemática colección y explicación de todas las palabras (o más estrictamente, unidades léxicas) de un lenguaje. 

\textbf{Resolución de la ambigüedad léxica}, siendo un ejemplo la determinación del sustantivo concreto dada una referencia indirecta.

\textbf{Aprendizaje automático en PLN}, es decir la aplicación de algoritmos de aprendizaje no supervisado para sistemas de esta índole.

\textbf{Generación textual monolingüe y multilingüe}. Esto es, la expresión en lenguaje natural de información estructurada, permitiendo a un ordenador generar a tiempo real respuestas más accesibles a un usuario.

\textbf{Traducción automática}, refiriéndonos a sistemas capaces de realizar la traducción de textos naturales a otro lenguaje natural con poca o nula supervisión humana.

\textbf{Reconocimiento y síntesis del habla}, atendiendo a problemática como la determinación de palabras homófonas a través del contexto.

\textbf{Extracción y recuperación de información monolingüe, multilingüe y multimodal}, en la que se aprovecha información multimedia recibida a tiempo real para analizar la intención del emisor.

\textbf{Sistemas de búsqueda de respuestas}, alternativas a los históricos sistemas expertos, en los que se pretende responder a preguntas del usuario sobre un ámbito reducido.

%\textbf{Análisis automático del contenido textual}.

\textbf{Resumen automático}, por el cual obtenemos el resumen de un texto intentando conservar la información relevante.

%\textbf{PLN para la generación de recursos educativos}.

\textbf{PLN para lenguas con recursos limitados}, como es el caso de lenguas indígenas o autóctonas.

\textbf{Aplicaciones industriales del PLN}, entre las más importantes la generación de información para la toma de decisiones en empresas, o Inteligencia Empresarial (\textit{Business Intelligence}).

\textbf{Sistemas de diálogo}, sistemas que se comunican con el usuario utilizando lenguaje natural, de manera similar a mantener una conversación. 

\textbf{Análisis de sentimientos y opiniones}, actualmente un área del PLN con gran popularidad y que puede considerarse como una de sus aplicaciones industriales de mayor interés.

\textbf{Minería de texto}, el procesamiento estadístico de grandes cantidades de texto con la intención de extraer información de alta calidad.

%\textbf{Evaluación de sistemas de PLN}.

\textbf{Implicación textual y paráfrasis}, con la que podemos visualizar información textual de diversas maneras, así como conocer las enunciaciones que se derivan de ellas.

Se tuvieron en cuenta estas áreas temáticas a la hora de decidir el enfoque que tendría el trabajo, tomando en cuenta variables como factibilidad, tiempo disponible y utilidad, entre otras.



\section{Ámbito del problema}

Se decidió realizar un trabajo sobre Procesamiento de Lenguaje Natural, siendo este el primer punto acordado antes de proceder a la definición del problema a tratar para la realización del trabajo.

\section{Definición del problema}

El caso de estudio escogido para este trabajo puede ser subdividido en dos problemas: el primero es la obtención del texto a clasificar, ya sea desde un archivo de texto en local o a través de Internet.

El segundo problema es el más importante y el cual es objetivo principal de este trabajo: el procesamiento del texto una vez obtenido.

Se ha decidido presentar las secciones de manera cronológicamente equivalente a la que se produce en cualquier problema de Procesamiento de Lenguaje Natural.

%% TODO : Does this go here?

%\subsection{Datos de trabajo}

\subsection{Obtención del texto}

Hoy en día existen muchas fuentes de las cuales obtener texto a la hora de realizar procesamiento de texto: se puede obtener de Internet, puede ser creado o incluso puede obtenerse de un medio físico como pueden ser los libros o la voz de un usuario.

La obtención del texto a procesar puede realizarse de muchas maneras, existiendo varias opciones para cada fuente de texto que se puede presentar. 

En el caso de Internet, existen tanto librerías para lenguajes de programación como programas dedicados a la extracción de texto. Normalmente funcionan de manera similar a un navegador de Internet, utilizando el protocolo HTTP para enviar peticiones a servidores web y recibir contenidos de las páginas web, que luego son desglosados mediante \textit{parsers} para obtener los datos deseados.

En nuestro caso, el objetivo es la clasificación de los Trabajos de Fin de Grado. El Repositorio Institucional de la Universidad de La Laguna \cite{riull} es un Servicio Web que contiene una gran cantidad de documentos relacionados con la actividad docente de la Universidad, entre ellos los Trabajos de Fin de Grado.

\subsection{Procesamiento del texto}

El procesamiento de texto es la parte del problema que toma una mayor importancia, ya que es la más complicada y objeto de este trabajo.

Para nuestro caso de estudio, nos concentramos en un subproblema del campo; la clasificación de textos. La clasificación de textos es una de las de mayor relevancia actual.

\lhead{\emph{\ChapterOne{}}}
El estudio previo de las opciones a considerar a la hora escoger las herramientas para realizar el proyecto ha ocupado una gran parte del tiempo dedicado.

Esto es en parte debido a la gran cantidad de herramientas disponibles para la tarea a realizar: existen decenas de herramientas para el Procesamiento del Lenguaje Natural, y cientos para la extracción de texto a partir de fuentes en Internet.

Para realizar el trabajo se han considerado varias herramientas, teniendo en cuenta puntos como la facilidad de uso, la adaptabilidad de la herramienta, su capacidad de procesamiento, las posibilidades de ser utilizada con entradas de texto en español, etc.

\section{Herramientas de PLN}

Se consideró un número de herramientas para el Procesamiento del Lenguaje Natural, las cuales fueron evaluadas según las ventajas y desventajas que proporcionarían al desarrollo del proyecto teniendo en cuenta los objetivos que se deseaban alcanzar.

La herramienta utilizada en este caso es la biblioteca NLTK (\textit{Natural Language Toolkit}) desarrollada en el lenguaje de programación Python, pero es necesario conocer las decisiones que llevaron a esta conclusión.

A continuación se muestran las herramientas consideradas para el Procesamiento del Lenguaje Natural en la realización de este trabajo.

\subsection{Stanford CoreNLP}

Se trata de una herramienta de Procesamiento de Lenguaje Natural implementada en Java por la Universidad de Stanford \cite{stanford-corenlp-paper}.
%
%(Ha sido implementada en Java, y la ...entrega... actual requiere Java 1.8+).
%
Fue una de las herramientas de mayor importancia a considerar, debido a la familiaridad que se tenía con la metodología de trabajo a la hora de utilizarlo.
%
Actualmente cuentan con modelos lingüísticos para el chino, inglés, francés, alemán, y español, el idioma objetivo de este trabajo.
%
Una de las principales ventajas de esta herramienta es su implementación en Java, un lenguaje de más bajo nivel que Python o Ruby, lo que le proporciona una mayor capacidad de cómputo de datos.

\subsection{Apache Lucene y Solr}

Una de las grandes ventajas del sistema Solr es que posee una API (\textit{Application Programming Interface}) flexible con la que se puede interactuar mediante los lenguajes de programación Ruby y Python, así como también es posible comunicarnos mediante el paso de mensajes JSON (JavaScript Object Notation), de manera similar a muchas arquitecturas servidor/cliente en la red \cite{apache-lucene}.

\subsection{Apache OpenNLP}

Esta herramienta utiliza una aproximación diferente a la que usa el Stanford CoreNLP.
%
Se trata de un sistema desarrollado en Java, que permite ser utilizado como una biblioteca Java. Además de esto, posee una interfaz de programación en línea (\textit{scripting}) que puede también usarse mediante la línea de comandos \cite{apache-open-nlp}.
%
A pesar de encontrarse algo desfasada en comparación con el resto de opciones, permanece como una opción robusta y rápida de implementar.

%\subsection{Apache UIMA}

%Se observó otra herramienta Apache para el Procesamiento de Lenguaje Natural...

%\subsection{FreeLing}
%asd

\subsection{GATE}

El GATE (\textit{General Architecture for Text Engineering}) es un sistema de software libre con una gran capacidad de procesamiento de texto \cite{gate}.
%
Fue desarrollado inicialmente por un equipo base de 16 programadores \cite{gate-about}, empezando en 1995 como parte de un proyecto del EPSRC (Engineering and Physical Sciences Research Council), una organización basada en el Reino Unido con el objetivo de financiar la investigación científica en el país.
%
Dado el largo tiempo que lleva esta herramienta en desarrollo, que abarca más de dos décadas, cuenta con un amplio abanico de funcionalidades de procesamiento de texto.

\subsection{NLTK, Natural Language Toolkit}

El NLTK (\textit{Natural Language Toolkit}) es una biblioteca de Procesamiento de Lenguaje Natural que utiliza el lenguaje de programación Python \cite{nltk-book}.
%
NLTK es software libre, lo que permite a estudiantes y al personal académico realizar estudios con la herramienta sin necesidad de realizar una inversión económica.
%
Esta herramienta es también de código abierto, lo que lo hace ideal para expandir sus funcionalidades en caso de necesitarlo.
%
El hecho de estar implementada como una biblioteca Python reduce la curva de aprendizaje, y la acerca al mundo académico, cuya mayor parte de integrantes se encuentra familiarizado con este lenguaje de programación.

\label{chosen-nlp-tool}
\section{Herramienta de PLN elegida}

Una razón para la elección de NLTK como herramienta es el gran soporte que tiene, debido a las dimensiones de su comunidad de usuarios. Es una de las herramientas de Procesamiento de Lenguaje Natural de mayor aceptación en el ámbito científico.

Otra razón importante es el apoyo que proporciona el libro \textit{Natural Language Processing with Python}\cite{nltk-book}. Este libro tiene por autores a los creadores del NLTK, haciéndolo idóneo para comprender y utilizar todas las funcionalidades que aporta esta biblioteca.

\section{Lenguaje de programación escogido}

El código de las herramientas de Procesamiento de Lenguaje Natural desarrolladas en Java y C++ es compilado a un código máquina de bajo nivel, lo que le confiere una mayor rapidez de cómputo.

Ésta amplia capacidad de procesamiento en comparación a herramientas para Python o Ruby es aprovechable ya sea en aumentar la cantidad de datos a procesar o reducir el tiempo que lleva el procesamiento de los mismos.

Sin embargo, el que una herramienta esté diseñada para un lenguaje de éstas características tiene como consecuencia la reducción de flexibilidad inmediata de la herramienta, ya que el desarrollo de software en los lenguajes fuertemente tipados conlleva una mayor cantidad de tiempo a la hora de especificar la relación, jerarquía y tipado de los datos, con el fin de aprovechar el aumento en velocidad que ofrecen.

Es por esta razón que se optó por lenguajes de guiones (\textit{scripting}), como Python o Ruby, que sin perder generalidad en cuanto a la resolución de problemas, y a pesar de tener una velocidad de cómputo más reducida, permiten el desarrollo de soluciones adaptadas a los problemas que surgen con una mayor flexibilidad y rapidez, ofreciendo además una mayor cantidad de funcionalidad estándar para el procesamiento de texto.

Se decidió utilizar Python para complementar la elección de herramienta de PLN que se hizo, en este caso la biblioteca NLTK para Python, como se mencionó anteriormente (\ref{chosen-nlp-tool}).


---

A continuación, en el Capítulo 2, se expone la problemática concreta del caso de estudio así como las soluciones implementadas con las herramientas escogidas. El Capítulo 3 expone los resultados alcanzados y realiza una comparativa, mientras que el Capítulo 4 extrae conclusiones sobre el trabajo realizado, y plantea futuras líneas de trabajo. Finalmente, el Capítulo 5 desglosa los recursos utilizados en la elaboración de este proyecto.