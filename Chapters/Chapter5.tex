\chapter{\ChapterFive{}}
%\lhead{\emph{\ChapterFive{}}}

\setstretch{1.5}  % Set the line spacing to 1.5, this makes the following tables easier to read

Aquí se expone el presupuesto equivalente al trabajo realizado. No se produjeron gastos de infrastructura puesto que los servicios de Internet y red eléctrica utilizados fueron el doméstico y el de dominio público. No se dan costes de equipo puesto que el ordenador utilizado fue el de uso personal, mientras que el software utilizado es libre y por lo tanto no se incurrió en gastos.

\begin{table}[!ht]
\centering
\caption{Presupuesto}
\label{tabla-presupuesto}
\begin{tabular}{|l|l|l|l|}
\hline
Descripción      & Costo por unidad & Cantidad & Costo total \\ \hline
Horas de trabajo & 9\euro{}         & 300      & 2700\euro{} \\ \hline
Total:           & ---              & ---      & 2700\euro{} \\ \hline
\end{tabular}
\end{table}