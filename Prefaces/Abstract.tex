% The Abstract Page
\addtotoc{Abstract}  % Add the "Abstract" page entry to the Contents
%\abstract{
\addtocontents{toc}{\vspace{1em}}  % Add a gap in the Contents, for aesthetics

\bigskip
%\center{\huge{\textit{Abstract}} \par}
\begin{center}
\huge{\textit{Abstract}} \par
\end{center}

%The aim of this work is to study the currently available tools for Natural Language Processing that might be applied in study cases for the Spanish language, as most of these tools nowadays focus on the English language.

The aim of this work is the development of a study case in Spanish in which Natural Language Processing techniques are applied, to observe the effectiveness of the NLP tools available for this language.

The study case chosen for this work is the classification of the Final Degree Projects made avaliable to the public in the Institutional Repository of the University of La Laguna.
%}
\vskip 40pt
\textbf{Keywords:} \textit{NLP, Natural Language Processing, NLTK, Natural Language Toolkit, Python, CasperJS, JavaScript}

%\bigskip
\clearpage
\addtotoc{Resumen}
%\center{\huge{\textit{Resumen}} \par}
\begin{center}
\huge{\textit{Resumen}} \par
\end{center}

%La finalidad de este trabajo es el estudio de las herramientas actualmente disponibles para el Procesamiento del Lenguaje Natural que puedan ser aplicadas a casos de estudio para el idioma español, dado que la mayoría de estas herramientas hoy en día se centran en el inglés.

La finalidad de este trabajo es el desarrollo de un caso de estudio en lenguaje español sobre el cual aplicar técnicas de Procesamiento del Lenguaje Natural, para constatar la eficacia de las herramientas PLN disponibles para este idioma.

El caso de estudio escogido para desarrollar en este trabajo fue la clasificación de los Trabajos de Fin de Grado puestos a disposición del público en el Repositorio Institucional de la Universidad de La Laguna.

\vskip 40pt
\textbf{Palabras clave:} \textit{PLN, Procesamiento del Lenguaje Natural, NLTK, Natural Language Toolkit, Python, CasperJS, JavaScript}
%}

\clearpage  % Abstract ended, start a new page
%% ----------------------------------------------------------------