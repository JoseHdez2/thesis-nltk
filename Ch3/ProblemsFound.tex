\lhead{\emph{\ChapterThree{}}}
\subsection{Obtención del texto}

Uno de los problemas a la hora de realizar la extracción de texto en la página del Repositorio Institucional de la ULL se producía cuando se realizaba un número excesivo de consultas a la página. Esto provocaba que las peticiones enviadas desde la IP del ordenador de pruebas fuesen ignoradas por el servidor. Se puede conjeturar que esto sea resultado de un mecanismo de defensa del servidor institucional para contrarrestar ataques DoS (\textit{Denial of Service}). 
\todo{
Esto provocaba una interrupción en el flujo de trabajo, impidiendo la obtención de datos hasta 
}

\subsection{Clasificación del texto}

% Para la clasificación de los textos DURANTE LA PRIMERA APROXIMACION
Para la clasificación de los textos, uno de los problemas más importantes fue el exceso de clases en comparación con el número de instancias. Muchas de estas clases contenían a un único elemento, lo que dificultaba el proceso de clasificación al no obtener suficiente información determinante para caracterizar cada tipo de clase.

Como se comentó en el apartado \ref{classification-conclusions}, en muchos casos, la descripción o abstracto del TFG no se encuentra como metadato de los TFG de la RIULL, debido a que no es obligatorio rellenar este campo para dar de alta un TFG en la plataforma.
%
Esto redujo drásticamente la capacidad de clasificación sobre los metadatos de los TFGs en comparación con el caso en que todos los documentos tuviesen un resumen o abstracto asignado en el servicio.

\subsection{Realización de la memoria}

Por defecto \textit{LaTeX} no permite el uso de tildes y caracteres españoles, por lo que fue necesario incluir el paquete \texttt{inputenc} y \texttt{babel} en la configuración del proyecto, con las opciones \texttt{utf8} y \texttt{spanish}, respectivamente.

A la hora de mostrar el código del trabajo realizado, el primer método considerado fue la realización de capturas del código en el editor de texto, para insertarlas en el documento como imágenes.
Esto conllevaba una falta de uniformidad, debido al diferente tamaño de las imágenes, así como dificultad a la hora de cambiar la apariencia de las mismas.

Como alternativa, se utilizó un paquete \textit{LaTeX} llamado \texttt{listings}, que permite la introducción de código con formato, y proporciona funcionalidad que ayuda a su lectura como el resaltado (\textit{highlighting}) de palabras reservadas.
%
Tuvo que ser modificado para permitir el resaltado de código JavaScript, cuya sintaxis no se incluía por defecto, para resaltar los listados apropiados.